\documentclass[12pt]{article}
\usepackage[margin=2.5cm]{geometry}
\usepackage[spanish]{babel}
\usepackage{graphicx}
\usepackage{parskip}
\usepackage{array}
\renewcommand{\baselinestretch}{1.25}
\pagestyle{empty}

\begin{document}

% === PÁGINA 1: RESOLUCIÓN PRINCIPAL ===
\begin{minipage}{0.3\textwidth}
  \vspace{-0.6cm}
  % RUTA CORREGIDA: Le decimos que suba un nivel (../) y entre a la carpeta de recursos.
  \includegraphics[width=2.5cm]{../05_Templates_y_Recursos/logo.png}
\end{minipage}
\hfill
\begin{minipage}{0.65\textwidth}
  \begin{flushright}
    Buenos Aires, {{ fecha_larga }}
  \end{flushright}
\end{minipage}

\vspace{1cm}

\begin{center}
  \textbf{\large Resolución {{ codigo_res }} – {{ titulo_documento }}}
\end{center}

\vspace{0.8cm}

\textbf{VISTO:}
{{ visto }}

\vspace{0.4cm}

\textbf{CONSIDERANDO:}
\begin{itemize}
  \item Que la mensualidad correspondiente al mes de {{ mes_anterior }} fue de \textbf{\${{ mensualidad_anterior_monto }} ARS}.
  \item Que dicho monto fue utilizado en los siguientes conceptos:
    \begin{itemize}
      
      \item {{ gasto.descripcion }}: \textbf{\${{ gasto.monto }}}
      
    \end{itemize}
  
  \item {{ item }}
  
\end{itemize}

\vspace{0.4cm}

\textbf{RESUELVO:}

\textbf{ARTÍCULO {{ loop.index }}°.-}
{{ articulo }}
\vspace{0.5cm}


\vspace{1.4cm}

\begin{flushright}
  % RUTA CORREGIDA: También para la firma.
  \includegraphics[width=3.2cm]{../05_Templates_y_Recursos/firma.png}
\end{flushright}

% === PÁGINA 2: ANEXO I (SI EXISTE) ===

\newpage

\begin{center}
  \textbf{\large Anexo I – {{ anexo.titulo }}}
\end{center}

\vspace{1cm}

\begin{tabular}{|p{8cm}|r|}
\hline
\textbf{Categoría} & \textbf{Monto (ARS)} \\
\hline

{{ item.categoria }} & \${{ item.monto }} \\

\hline
\textbf{Subtotal} & \textbf{\${{ anexo.subtotal }}} \\
\hline

{{ anexo.penalizacion.descripcion }} & - \${{ anexo.penalizacion.monto }} \\
\hline
\textbf{Total solicitado} & \textbf{\${{ anexo.total_solicitado }}} \\

\hline
\end{tabular}

\vspace{1cm}

\textit{Nota: {{ anexo.nota_final }}}

\vspace{2cm}

\begin{flushright}
  % RUTA CORREGIDA: Y para la firma del anexo.
  \includegraphics[width=3.2cm]{../05_Templates_y_Recursos/firma.png}
\end{flushright}


\end{document}